\documentclass[11pt]{article}

%%%%%%%%%%%%%%%%%%%%%%%%%%%%%%%%%%%
%          				PACKAGES			              	  %
%%%%%%%%%%%%%%%%%%%%%%%%%%%%%%%%%%%

% Sets margins
%\usepackage[left=.7in,top=.7in,right=.7in,bottom=.7in]{geometry} % Document margins
\usepackage[left=1in,top=1in,right=1in,bottom=0.5in]{geometry} % Document margins

%% additional packages
\usepackage{array}
\newcolumntype{L}[1]{>{\raggedright\let\newline\\\arraybackslash\hspace{0pt}}m{#1}}
\newcolumntype{C}[1]{>{\centering\let\newline\\\arraybackslash\hspace{0pt}}m{#1}}
\newcolumntype{R}[1]{>{\raggedleft\let\newline\\\arraybackslash\hspace{0pt}}m{#1}}
\usepackage{booktabs}
\usepackage{threeparttable}
\usepackage[labelsep=period]{caption}
\usepackage{setspace}
\usepackage{amsfonts}
\usepackage{graphicx}
\usepackage{xcolor}
\usepackage[colorlinks = true,
            linkcolor = blue,
            urlcolor  = blue,
            citecolor = blue,
            anchorcolor = blue]{hyperref}

\newcommand{\Dochref}[3][blue]{\href{#2}{\color{#1}{#3}}}%

\usepackage{fancyhdr}
\pagestyle{fancy}
\usepackage{lastpage}
\fancyhf{} % sets both header and footer to nothing
\renewcommand{\headrulewidth}{0pt}
\usepackage{sectsty}
\usepackage{pdfpages}

%% BibTeX settings
\usepackage{natbib}
\bibliographystyle{apalike}
%\bibliographystyle{unsrtnat}
\bibpunct{(}{)}{,}{a}{,}{,}


% Removes indentation from first line of paragraphs
\usepackage[parfill]{parskip}

%% markup commands for code/software
\let\code=\texttt
\let\pkg=\textbf
\let\proglang=\textsf
\newcommand{\file}[1]{`\code{#1}'}
\newcommand{\email}[1]{\href{mailto:#1}{\normalfont\texttt{#1}}}
\urlstyle{same}

%-----------------------------------------------------------------------------------%


% Options
\singlespacing
%\usepackage{times}

 \usepackage{pdfpages}

\begin{document}


%-----------------------------------------------------------------------------------%
% ADDRESS %
%-----------------------------------------------------------------------------------%


\begin{centering}
{\bf \Large Econ 4676: Big Data and Machine Learning \\ for Applied Economics}\\
{\bf \Large Syllabus (Preliminar)}\\
{\bf  \href{https://ignaciomsarmiento.github.io/}{Ignacio Sarmiento-Barbieri}}

\end{centering}


\bigskip

{\bf \Large  Información del Curso}
\medskip

Clases: Virtuales, Martes y Jueves de 11:00 a 12:30

Sitio web: \url{https://github.com/ECON-4676-UNIANDES}

Horario de Oficina: A concretar vía  correo electrónico.

email: \href{mailto:i.sarmiento@uniandes.edu.co}{i.sarmiento@uniandes.edu.co}

\bigskip

{\bf \Large  Descripción General}

\medskip

El objetivo  de  este  curso  es  introducir  a  los  alumnos a  un  conjunto  de  herramientas  estadísticas, matemáticas, y computacionales para abordar problemas de gran cantidad/tipos/calidad de datos (“large n”), y cantidad de variables (“large p”). Problemas de predicción e inferencia, con especial énfasis en inferencia causal, atravesarán transversalmente al curso. Se buscará también familiarizar a los alumnos con la literatura reciente que utiliza estas herramientas. Mediante una combinación de  conjuntos  de  talleres,  presentaciones, exámenes, y  un  trabajo  final  grupal,  los  estudiantes  adquirirán  las herramientas  estadísticas  y  computacionales  necesarias  para  hacer  uso  de  big  data  y  machine learning en investigación empírica.




\bigskip

{\bf \Large Prerrequisitos}

\medskip
Microeconomía 3 y Econometría 1. Se recomienda experiencia con programación en {\texttt R}, aunque no es requisito. Si bien no es requisito tener experiencia con {\texttt R} si es requisito {\it tener mucha voluntad de aprender y experimentar}. Este programa (y todos) se aprende utilizándolos!


\bigskip


{\bf \Large Evaluación}
\medskip
\begin{itemize}
	\item 10\% Participación
	\item 40\% Talleres
	\item 25\% Propuesta de trabajo
	\item 25\% Examen Final
\end{itemize}


{\bf Participación}. La participación de los estudiantes es fundamental para sacar el mayor provecho del curso. La virtualidad impone nuevos desafíos y es importante mantenerse conectados para crear las sinergias que surgen de las interacciones humanas. Si bien participación es la actividad con menos peso en la composición final, será el {\it ``tiebreaker''} por el cual decidiré la nota final. Participación no incluye solamente la asistencia a clases, sino también actividades fuera de clase. Una vez registrados en el curso los estudiantes recibiran invitación al canal de \href{https://slack.com/}{Slack}, al aula virtual de \href{https://aws.amazon.com/education/awseducate/}{AWS} y a \href{https://github.com/ECON-4676-UNIANDES}{github}. La participación será juzgada en función a la participación en las discusiones, en los trabajos grupales, de las interacciones en el canal de \href{https://slack.com/}{Slack}, el aprovechamiento de \href{https://aws.amazon.com/education/awseducate/}{AWS} y se espera que estudiantes encuentren al menos un error de tipeo o cualquier otro tipo y los arregrlen a traves de {\it pull requests} en \href{https://github.com/ECON-4676-UNIANDES}{github}


{\bf Talleres}.  Los estudiantes realizarán trabajos prácticos grupales para evaluar su aprendizaje. Los groupos no podrán superar los 4 miembros. Habrán 4 talleres durante el semestre. Se dedicarán al menos 4 clases para la discusión y presentación de los talleres. Los talleres serán submitidos via \href{https://github.com/ECON-4676-UNIANDES}{github} y parte de nota de la particpación saldra de la evaluación de la historia del repositorio donde se verá la contribución de cada estudiante.


{\bf Propuesta de trabajo}. El producto final de este curso es un plan de trabajo con una propuesta de cómo implementar los conceptos y herramientas aprendidas a un problema concreto. La actividad es grupal y puede estar constituida por los mismos miembros del grupo de taller. La actividad estará dividida en 3 entregas. En la primera entrega los grupos se reunirán conmigo y presentaran brevemente (maximo 5 slides) la idea y como planean llevarla a cabo. En una segunda entrega donde se expondrán los datos propuestos. La entrega final será al concluir el curso que consolida toda el trabajo. Se otorgarán bonos a los estudiantes que además de presentar el plan de trabajo o propuesta, entreguen resultados concretos.

{\bf Examen final}. Este examen pretende evaluar los conceptos y habilidades aprendidas en el curso. Va a ser un examen domiciliario, de tiempo fijo, entre 48-72 horas.

\bigskip

{\bf \Large Libros y Recursos ({\it Preliminar y sujeto a cambios})}
\medskip

\begin{itemize}
	\item Farrell, D., Greig, F., and Deadman, E. (2020).
\newblock Estimating family income from administrative banking data: A machine
  learning approach.
\newblock {\em AEA Papers and Proceedings}, 110:36--41.


\item Glaeser, E.~L., Kominers, S.~D., Luca, M., and Naik, N. (2018).
\newblock Big data and big cities: The promises and limitations of improved
  measures of urban life.
\newblock {\em Economic Inquiry}, 56(1):114--137.


\item Hastie, T., Tibshirani, R., and Friedman, J. (2009).
\newblock {\em The elements of statistical learning: data mining, inference,
  and prediction}.
\newblock Springer Science \& Business Media.

\item Hastie, T., Tibshirani, R., and Wainwright, M. (2015).
\newblock {\em Statistical learning with sparsity: the lasso and
  generalizations}.
\newblock CRC press.


\item James, G., Witten, D., Hastie, T., and Tibshirani, R. (2013).
\newblock {\em An introduction to statistical learning}, volume 112.
\newblock Springer.

\end{itemize}

\bigskip

{\bf \Large  Temario ({\it Preliminar y sujeto a cambios}) }
\medskip

\begin{enumerate}
 	\item Introduction to ML: prediction and inference. Supervised and unsupervised learning. MCO revision. Goodness of fit. Introduction to R, Jupyter Lab, Github, and AWS
	\item  New Economic Observation. Search and computer-mediated behavior. Text Data: news media and social media. Large N Problems: compute and processing. Web scrapers and APIs.
	\item Observing from above: Introduction to spatial econometrics. Modeling spatial dependence. Processing big spatial/satellite data, raster data.
	\item Intro to non parametric econometrics. Kernels, densities, and non parametric regressions. The curse of dimensionality.
	\item Classification: Bayes Risk, Logit Models, ROC analysis.
	\item Non lineal methods: Clusters, PCA, K-means, Tress, Boosting andRandom Forests, Support vector machines.
	\item Bonus Track: Machine Learning for Causal Inference and Deep Learning.
\end{enumerate}

\end{document}
